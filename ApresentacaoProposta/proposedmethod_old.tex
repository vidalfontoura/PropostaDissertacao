%%===================Método proposto==============%%
\section{Método Proposto}
\subsection{Método Proposto}
\frame{
  \frametitle{Método Proposto}
   
   \begin{block}{Escolha dos algoritmos multiobjetivos}
      	\begin{itemize}
      	 \item NSGA-II (2002)
      	 \item SPEA2 (2001)
      	\end{itemize}
   \end{block}
   
   \begin{block}{Proposta}
      Utilização de técnicas de hiper heurística nos algoritmos selecionados, afim de avaliar as técnicas de Choise Function e Random na seleção de casos de teste.
    \end{block}
}

\frame{
	\frametitle{Abordagens propostas}
	
	\begin{columns}[c]
		\column{0.7\textwidth}
		\SetAlFnt{\tiny}
	
		\begin{algorithm}[H]
			\caption{Pseudocódigo do NSGA-II}
			\label{alg:nsgaII}
			\Entrada{$N$, $g$}
			\Saida{Melhor conjunto de soluções para o problema}
			\Inicio{
				Inicialização e avaliação da população $P$ de tamanho $N$\;
				\Para{$t \leftarrow 0$ até $g$}{
					Inicializa a população de filhos $Q$ de tamanho $N$\;
					\Para{$i \leftarrow 0$ até $\frac{N}{2}$}{
						\colorbox{gray!40}{Seleção da melhor hiper heurística}\\
						Seleção dos pais\;
						\colorbox{gray!40}{Aplicação da melhor hiper heurística}\\
						Avalia-se os indivíduos criados\;
						Guarda os indivíduos criados em $Q$\;
						$t = t + 2$\;
						\colorbox{gray!40}{Atualiza o \textit{ranking}}\\
						\colorbox{gray!40}{Atualiza o \textit{elapsed time}}\\
					}
					$R \leftarrow P \cup Q$\;
					Aplica-se \textit{Fast Non-Dominated Sorting} em $R$ gerando vetores $F_{i}$ de soluções não dominadas\;
					Calcula-se o \textit{Crowding Distance Sorting} de cada solução em $F_{i}$\;
					Popula $P$ com as soluções dos melhores vetores $F$\;
					
				}	
				\Retorna{Retorna as soluções não dominadas de $P$}
			}	
		\end{algorithm}
	\end{columns}
}

\frame{
	\frametitle{Abordagens propostas}
	
	\begin{columns}[c]
		\column{0.7\textwidth}
		\SetAlFnt{\tiny}
		
		\begin{algorithm}[H]
			\caption{Pseudocódigo do SPEA2}
			\label{alg:spea2}
			\Entrada {$pop$, $t_{c}$, $t_{m}$, \textit{critério de parada}}
			$ND \leftarrow $ Solução Inicial()\;
			$P_{0} \leftarrow $ Solução Inicial Aleatória($pop$, $ND$)\;
			$A_{0} \leftarrow 0$\; 
			$t \leftarrow 0$\;
			\Enqto{\textit{critério de parada} não satisfeito}{
				\ParaCada{$i \in P_{t} \cup A_{t}$}{
					\ParaCada{$j \in P_{t} \cup A_{t}$}{
						\Se{$i \succ j$}{
							$S(i) = S(i) + 1$\;
						}
					}
					\ParaCada{$j \in P_{t} \cup A_{t}$}{
						\Se{$i \succ j$}{
							$R(i) = R(i) + S(j)$\;
						}
					}
					$D(i) \leftarrow \dfrac{1}{\sigma^{k}_{i} + 2}$\; 
					$F(i) \leftarrow R(i) + D(i)$\;
				}
				$A_{i+1} \leftarrow$ Seleção Populacional ($P_{t} \cup A_{t}$, $N$)\;
				\Se{\textit{critério de parada} não satisfeito}{
					$P_{t+1} \leftarrow$ Gerar Indivíduos da Próxima Geração ($A_{t+1}$, $t_{c}$, $t_{m}$)\;
				}
				$t \leftarrow t + 1$\;
			}
			$ND \leftarrow $ soluções não-dominadas obtidas de $A_{t}$\;
			\Retorna $ND$\;
		\end{algorithm}
	\end{columns}
}

\subsection{Experimentos e Análise dos Resultados}
\frame{
	\frametitle{Experimentos e Análise dos Resultados}
	\begin{block}{Experimetos (Instâncias)}
		\begin{itemize}
		 \item Cas
		 \item James
		 \item Save
		\end{itemize}
	\end{block}
	
	\begin{block}{Análise dos Resultados}
		\begin{itemize}
			\item Escore de Mutação
			\item Número de Casos de Teste
			 \item Comparação dos algoritmos NSGA-II e SPEA2% (MOEA/DD e NSGA-III)
			 \item Utilizando as métricas: R2, Hypervolume, GD e IGD.
			 \item E o teste estatístico de Kruskall Wallis
		\end{itemize}
	\end{block}
}

\frame{
	\frametitle{Configuração dos Experimentos}
	\begin{block}{}
		\begin{itemize}
			\item População: 200
			\item Gerações: 600
			\item Seleção: Torneio Binário
		\end{itemize}
	\end{block}
	
	\begin{block}{Composição das heurísticas de baixo nível}
			\begin{table}[!hbt]
				\centering
				\begin{tabular}{|c|c|c|c|}
					\hline
					\multicolumn{1}{|l|}{} & \multicolumn{3}{c|}{\textbf{Mutação}} \\ 
					\hline 
					\textbf{Crossover} & - & Bit Flip & Swap Binary \\ 
					\hline
					Single Point & h1 & h2 & h3 \\ 
					\hline
					Uniform & h4 & h5 & h6 \\
					\hline
					Two Points  & h7 & h8 & h9 \\
					\hline
				\end{tabular}
			\end{table}
	\end{block}
}