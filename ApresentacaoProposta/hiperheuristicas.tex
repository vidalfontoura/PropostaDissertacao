%%========================Introducao================================%%
\section{Hiper Heurísticas}

%%========================Objetivo================================%%
\subsection{Hiper Heurísticas}
\frame{
	\frametitle{Hiper Heurísticas}
	
	\begin{block}{}
		\begin{itemize}
			\item   Apesar do sucesso de meta-heurísticas e outros métodos de busca ainda existem dificuldades de generalização destas estratégias. 
			\item  Esta dificuldade surge  da necessidade de selecionar
			os parâmetros e configurações mais adequados dos algoritmos para um problema ou instância.
			\item  Questão: \textbf{É possível
			automatizar o projeto e parametrização de métodos heurísticos para resolver problemas
			de busca computacional difíceis?}
			\item A ideia principal é desenvolver algoritmos que sejam
			mais genéricos do que as implementações de metodologias atuais.
		\end{itemize}
		
		
	\end{block}
}

\subsection{Hiper Heurísticas}
\frame{
	\frametitle{Hiper Heurísticas}
	
	\begin{block}{}
		\begin{itemize}
		\item Operam sobre o
		espaço de busca de heurísticas que por sua vez operam sobre o espaço de busca de um problema.
		\end{itemize}
		
		
	\end{block}
}

\subsection{Classificação Hiper Heurísticas}
\frame{
	\frametitle{Classificação Hiper Heurísticas}
	
	\begin{block}{}
			\begin{figure}[!htb]
				\centering
				\includegraphics[scale=.4]{figuras/ClassificacaoHiperHeuristica.png}
				\caption{Classificação Hiper Heurísticas}
				\label{fig:classificacaoHH}
			\end{figure}
	\end{block}
}

\subsection{Níveis }
\frame{
	\frametitle{}
	
	\begin{block}{}
			
	\end{block}
}

